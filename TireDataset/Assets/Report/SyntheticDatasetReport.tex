\documentclass{article}
\usepackage{graphicx}
\usepackage{amsmath}

\title{Synthetic Tire Dataset Creation Report}
\author{Dmitry Tetkin}
\date{\today}

\begin{document}

\maketitle

\section{Introduction}
This report documents the process of creating a synthetic dataset for 10 000 tire images. The dataset was generated from a set of tire assets located in the directory \texttt{TireDataset/Assets/}. The final dataset was stored in the directory \texttt{/Assets/Dataset/}. The purpose of this synthetic dataset is to provide various tire images with different tread depths, sizes, and variations in patterns to facilitate training and testing for machine learning models.

\section{Dataset Generation Process}
The dataset was generated using the provided scripts located at \texttt{TireDataset/Assets/}. These scripts handle the creation of images by adjusting various tire parameters such as depth, size, and tread design. The images are then saved in the \texttt{Dataset} folder with filenames containing relevant metadata such as depth, size, and other characteristics of the tire.

\subsection{Tire Depth Adjustments}
For every tire image, the tread depth was modified, while maintaining the tread design to simulate different levels of wear and tear. The depth was varied to provide a range of conditions. For example, the depth could be set to 0.1 cm for one tire, and 0.5 cm for another. 

Examples of tire images with different depths:
\begin{itemize}
    \item Tire with depth 0.1 cm: \texttt{tire\_4\_0.1\_16\_1\_1.jpg} 
    \item Tire with depth 0.5 cm: \texttt{tire\_5\_0.5\_6\_1\_1.jpg} 
\end{itemize}

\begin{figure}[h!]
\centering
\includegraphics[width=0.45\textwidth]{/Users/dmitry057/Projects/Kolobok/TireDataset/Assets/Dataset/tire_4_0.1_16_1_1.jpg}
\caption{Tire with depth 0.1 cm}
\end{figure}

\begin{figure}[h!]
\centering
\includegraphics[width=0.45\textwidth]{/Users/dmitry057/Projects/Kolobok/TireDataset/Assets/Dataset/tire_5_0.5_6_1_1.jpg}
\caption{Tire with depth 0.5 cm}
\end{figure}

\subsection{Tire Image Examples}
Some sample tire images generated by the scripts are as follows:
\begin{itemize}
    \item \texttt{tire\_1\_0.8\_13\_1\_1.jpg}
    \item \texttt{tire\_1\_0.8\_21\_1\_1.jpg}

\begin{figure}[h!]
\centering
\includegraphics[width=0.45\textwidth]{/Users/dmitry057/Projects/Kolobok/TireDataset/Assets/Dataset/tire_1_0.8_13_1_1.jpg}
\caption{Sample Tire Image 1}
\end{figure}

\begin{figure}[h!]
\centering
\includegraphics[width=0.45\textwidth]{/Users/dmitry057/Projects/Kolobok/TireDataset/Assets/Dataset/tire_1_0.8_21_1_1.jpg}
\caption{Sample Tire Image 2}
\end{figure}

\section{Conclusion}
The synthetic dataset generated for tires contains images with a variety of tread depths and sizes. This dataset is essential for training models to detect and classify tire conditions based on tread depth and size. The scripts used to create these images ensure that the tread design remains consistent while allowing for variations in depth and size. This approach facilitates the creation of a robust dataset for various machine learning tasks related to tire wear and damage detection.

\end{document}